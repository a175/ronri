\documentclass[a4paper,12pt,draft]{amsart}
\usepackage{amsmath, amssymb, amsthm, url}
\begin{document}

\section{$\forall x \in A$, $P(x)=0$.}
単純に同値な命題:
\begin{itemize}
\item $\forall x \in A$, $P(x)=0$.
\item $\forall x \in A$に対し, $P(x)=0$.
\item $\forall x \in A$に対し, $P(x)=0$が成り立つ.
\item 全ての$x\in A$に対し, $P(x)=0$が成り立つ.
\item 全ての$x\in A$に対し, $P(x)=0$.
\item $x\in A$を満たす$x$に対し, $P(x)=0$が成り立つ.
\item $x\in A$を満たす$x$に対し, $P(x)=0$.
\item $x\in A$となる$x$に対し, $P(x)=0$が成り立つ.
\item $x\in A$となる$x$に対し, $P(x)=0$.
\item 各$x\in A$に対し, $P(x)=0$が成り立つ.
\item 各$x\in A$に対し, $P(x)=0$.
\item 任意の$x\in A$に対し, $P(x)=0$が成り立つ.
\item 任意の$x\in A$に対し, $P(x)=0$.
\item どの$x\in A$に対しても, $P(x)=0$が成り立つ.
\item どの$x\in A$に対しても, $P(x)=0$.
\item どの$x\in A$についても, $P(x)=0$が成り立つ.
\item どの$x\in A$についても, $P(x)=0$.
\item どの$x\in A$でも, $P(x)=0$が成り立つ.
\item どの$x\in A$でも, $P(x)=0$.
\item どの$x\in A$も, $P(x)=0$を満たす.
\item どんな$x\in A$に対しても, $P(x)=0$が成り立つ.
\item どんな$x\in A$に対しても, $P(x)=0$.
\item どんな$x\in A$についても, $P(x)=0$が成り立つ.
\item どんな$x\in A$についても, $P(x)=0$.
\item どんな$x\in A$でも, $P(x)=0$が成り立つ.
\item どんな$x\in A$でも, $P(x)=0$.
\item どんな$x\in A$も, $P(x)=0$を満たす.
\item $x\in A$ごとに, $P(x)=0$が成り立つ.
\item $x\in A$ごとに, $P(x)=0$.
\item 各$x\in A$ごとに, $P(x)=0$が成り立つ.
\item 各$x\in A$ごとに, $P(x)=0$.
\item $x\in A$はいつも$P(x)=0$を満たす.
\item $x\in A$ではいつも$P(x)=0$が成り立つ.
\item $x\in A$ではいつも$P(x)=0$.
\item $x\in A$である限り$P(x)=0$が成り立つ.
\item $x\in A$である限り$P(x)=0$.
\item $x\in A$を満たす限り$P(x)=0$が成り立つ.
\item $x\in A$を満たす限り$P(x)=0$.
\item $x\in A$を満たしている限り$P(x)=0$が成り立つ.
\item $x\in A$を満たしている限り$P(x)=0$.
\item $x\in A$が成り立つ限り$P(x)=0$が成り立つ.
\item $x\in A$が成り立つ限り$P(x)=0$.
\end{itemize}
$x\in A$を分けたもの:
\begin{itemize}
\item $A$の各元$x$に対し, $P(x)=0$が成り立つ.
\item $A$の各元$x$に対し, $P(x)=0$.
\item $A$の任意の元$x$に対し, $P(x)=0$が成り立つ.
\item $A$の任意の元$x$に対し, $P(x)=0$.
\item $A$のどの元$x$に対しても, $P(x)=0$が成り立つ.
\item $A$のどの元$x$に対しても, $P(x)=0$.
\item $A$のどの元$x$についても, $P(x)=0$が成り立つ.
\item $A$のどの元$x$についても, $P(x)=0$.
\item $A$のどの元$x$でも, $P(x)=0$が成り立つ.
\item $A$のどの元$x$でも, $P(x)=0$.
\item $A$のどの元$x$も, $P(x)=0$を満たす.
\item $A$のどんな元$x$に対しても, $P(x)=0$が成り立つ.
\item $A$のどんな元$x$に対しても, $P(x)=0$.
\item $A$のどんな元$x$についても, $P(x)=0$が成り立つ.
\item $A$のどんな元$x$についても, $P(x)=0$.
\item $A$のどんな元$x$でも, $P(x)=0$が成り立つ.
\item $A$のどんな元$x$でも, $P(x)=0$.
\item $A$のどんな元$x$でも, $P(x)=0$を満たす.
\item $A$のどんな元$x$でも, $P(x)=0$.
\item $A$の元$x$はいつも$P(x)=0$を満たす.
\item $A$の元$x$ではいつでも$P(x)=0$が成り立つ.
\item $A$の元$x$ではいつでも$P(x)=0$.
\end{itemize}
Implicationを使ったもの:
\begin{itemize}
\item $x \in A\implies P(x)=0$.
\item $x\in A$ならば, $P(x)=0$が成り立つ.
\item $x\in A$ならば, $P(x)=0$.
\item $x\in A$なら$P(x)=0$が成り立つ.
\item $x\in A$なら$P(x)=0$.
\item $x\in A$ならいつでも$P(x)=0$が成り立つ.
\item $x\in A$ならいつでも$P(x)=0$.
\item $x\in A$とすると, $P(x)=0$が成り立つ.
\item $x\in A$とすると, $P(x)=0$.
\item $x\in A$のときにはいつも$P(x)=0$が成り立つ.
\item $x\in A$のときにはいつも$P(x)=0$.
\item $x\in A$のとき$P(x)=0$が成り立つ.
\item $x\in A$のとき$P(x)=0$.
\end{itemize}
Implication/only if を使ったもの:
\begin{itemize}
\item $x\in A$となるのは, $P(x)=0$のときのみ.
\item $x\in A$となるのは, $P(x)=0$となるときのみ.
\item $x\in A$となるのは, $P(x)=0$を満たすのときのみ.
\item $x\in A$となるのは, $P(x)=0$が成り立つときのみ.
\item $x\in A$となるのは, $P(x)=0$のときに限る.
\item $x\in A$となるのは, $P(x)=0$となるときに限る.
\item $x\in A$となるのは, $P(x)=0$を満たすのときに限る.
\item $x\in A$となるのは, $P(x)=0$が成り立つときに限る.
\item $x\in A$を満たすのは, $P(x)=0$のときのみ.
\item $x\in A$を満たすのは, $P(x)=0$となるときのみ.
\item $x\in A$を満たすのは, $P(x)=0$を満たすのときのみ.
\item $x\in A$を満たすのは, $P(x)=0$が成り立つときのみ.
\item $x\in A$を満たすのは, $P(x)=0$のときに限る.
\item $x\in A$を満たすのは, $P(x)=0$となるときに限る.
\item $x\in A$を満たすのは, $P(x)=0$を満たすのときに限る.
\item $x\in A$を満たすのは, $P(x)=0$が成り立つときに限る.
\item $x\in A$が成り立つのは, $P(x)=0$のときのみ.
\item $x\in A$が成り立つのは, $P(x)=0$となるときのみ.
\item $x\in A$が成り立つのは, $P(x)=0$を満たすのときのみ.
\item $x\in A$が成り立つのは, $P(x)=0$が成り立つときのみ.
\item $x\in A$が成り立つのは, $P(x)=0$のときに限る.
\item $x\in A$が成り立つのは, $P(x)=0$となるときに限る.
\item $x\in A$が成り立つのは, $P(x)=0$を満たすのときに限る.
\item $x\in A$が成り立つのは, $P(x)=0$が成り立つときに限る.
\item $P(x)=0$に限り, $x\in A$.
\item $P(x)=0$のときに限り, $x\in A$.
\item $P(x)=0$となるときに限り, $x\in A$.
\item $P(x)=0$を満たすのときに限り, $x\in A$.
\item $P(x)=0$が成り立つときに限り, $x\in A$.
\item $P(x)=0$に限り, $x\in A$となる.
\item $P(x)=0$のときに限り, $x\in A$$となる.
\item $P(x)=0$となるときに限り, $x\in A$$となる.
\item $P(x)=0$を満たすのときに限り, $x\in A$$となる.
\item $P(x)=0$が成り立つときに限り, $x\in A$$となる.
\item $P(x)=0$に限り, $x\in A$を満たす.
\item $P(x)=0$のときに限り, $x\in A$$を満たす.
\item $P(x)=0$となるときに限り, $x\in A$$を満たす.
\item $P(x)=0$を満たすのときに限り, $x\in A$$を満たす.
\item $P(x)=0$が成り立つときに限り, $x\in A$$を満たす.
\item $P(x)=0$に限り, $x\in A$が成り立つ.
\item $P(x)=0$のときに限り, $x\in A$$が成り立つ.
\item $P(x)=0$となるときに限り, $x\in A$$が成り立つ.
\item $P(x)=0$を満たすのときに限り, $x\in A$$が成り立つ.
\item $P(x)=0$が成り立つときに限り, $x\in A$$が成り立つ.
\end{itemize}
対偶を取ったもの:
\begin{itemize}
\item $P(x)\neq 0\implies x\not\in A$.
\item $P(x)\neq 0$を満たすのは$x\not\in A$のときのみ.
\item $P(x)\neq 0$となるのは$x\not\in A$のときのみ.
\item $P(x)\neq 0$を満たすのは$x\not\in A$のときに限る.
\item $P(x)\neq 0$となるのは$x\not\in A$のときに限る.
\end{itemize}
全体を否定するもの:
\begin{itemize}
\item `$\exists x\in A$ such that $P(x)\neq 0$'ではない.
\item $P(x)\neq 0$を満たす$x\in A$は存在しない.
\end{itemize}


\section{$\exists x \in A$, $P(x)=0$.}
単純に同値な命題:
\begin{itemize}
\item $\exists x \in A$, $P(x)=0$.
\item $\exists x \in A$ such that $P(x)=0$.
\item $\exists x \in A$ s.t. $P(x)=0$.
\item $x\in A$が存在し, $P(x)=0$が成り立つ.
\item $x\in A$が存在し, $P(x)=0$.
\item ある$x\in A$が存在し, $P(x)=0$が成り立つ.
\item ある$x\in A$が存在し, $P(x)=0$.
\item ある$x\in A$に対して, $P(x)=0$が成り立つ.
\item ある$x\in A$に対して, $P(x)=0$.
\item ある$x\in A$について, $P(x)=0$が成り立つ.
\item ある$x\in A$について, $P(x)=0$.
\item 次の条件を満たす$x\in A$が存在する: $P(x)=0$.
\end{itemize}
語順が入れ替わっているもの:
\begin{itemize}
\item $P(x)=0$となる$x\in A$が存在する.
\item $P(x)=0$を満たす$x\in A$が存在する.
\item $P(x)=0$を満足する$x\in A$が存在する.
\item $P(x)=0$が成り立つ$x\in A$が存在する.
\item $P(x)=0$なる$x\in A$が存在する.
\end{itemize}
全体を否定するもの:
\begin{itemize}
\item `$\forall x \in A$, $P(x)\neq 0$'ではない.
\item `$x \in A\implies P(x)\neq 0$'ではない.
\end{itemize}

\end{document}

